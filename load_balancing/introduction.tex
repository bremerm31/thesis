% DGSWEM and the load balancing problem
This chapter examines the potential benefits of implementing DGSWEM, a discontinuous Galerkin (DG) finite-element storm surge code, on a task-based asynchronous execution model, and investigates various load balancing strategies for the resulting program.
One of the key aspects of DGSWEM is its ability to simulate coastal inundation during hurricane landfall.
The DG kernel can be implemented in a manner such that dry regions of the simulation require no computational work; however, as the hurricane inundates the coast, this optimization introduces significant dynamic load imbalance.
In order to address the load imbalance while still fitting the problem in machine memory, two constraints (load and memory) must be accounted for simultaneously.

While multi-constraint graph partitioning tools have been used to obtain good static partitions for these scenarios, they perform sub-optimally for irregular applications such as ours.
%On the other hand, to address the dynamic nature of the simulation, dynamic load balancing strategies have been studied extensively.
%However, these approaches typically are based on single constraint partitioning strategies.
On the other hand, fully dynamic load balancing strategies are typically based on balancing a single constraint, which is insufficient for hurricane simulation.
In order to overcome these shortcomings, we investigate dynamic and semi-static multi-constraint load balancing approaches that can be leveraged to run hurricane simulations efficiently on future supercomputing platforms.

% our approach
To circumvent the software engineering effort to port DGSWEM to a task-based model and to allow extremely rapid evaluation of the resulting execution under a variety of parameters (e.g. choice of load balancing algorithm), we have developed DGSim, a task-based execution model and discrete-event simulation tool that features asynchronously migratable, globally addressable objects.
DGSim was designed to natively model lightweight, one-sided, active interprocess messaging along with distributed, asynchronously migratable objects.  Together, these allow the simulation to fully overlap both the computation of the new tile placements and the resulting movement of the tile data with the application's main computation.  These features are not readily available in other simulation tools but are key features of the fully asynchronous load balancing algorithms introduced here.
The development of DGSim allows us to estimate the performance of hurricane simulations running with these advanced features in a variety of configurations.
% DGSim is additionally able to mimic mechanisms that are common across production frameworks, such as automated task scheduling, load balancing, and data movement.
Crucially, DGSim allows us to simulate the computational behavior of DGSWEM in an asynchronous task-based runtime using skeletonization, eliminating the need to execute the costly computational kernels, resulting in a lightweight approach that allows for rapid algorithm prototyping and evaluation.
The main contributions of this section are:
\begin{enumerate}
\item The development of multi-constraint dynamic load balancing strategies specifically geared for the irregularity associated with the simulation of hurricane storm surge. In doing so, we present a dynamic and semi-static algorithm.
\item The \emph{trellis} approach for semi-static repartitiong strategies which fully leverages the asynchronous nature of the run-time. This approach can be extended to ensure efficient semi-static load balancing for a wide range of problems.
\item The development and validation of DGSim, a discrete-event simulator that enables rapid prototyping and evaluation of {\em fully asynchronous} load balancers for task-based parallel programs.
\end{enumerate}

%outline of paper
The outline of the chapter is as follows. Section~\ref{sec:related} presents related work.  In Section \ref{sec:dgswem}, we discuss the DG kernel and the irregular nature of the inundation problem. Thereafter, we outline DGSim, which simulates the performance of an asynchronous task-based implementation of DGSWEM. Section \ref{sec:balancers} presents formalism for load balancing in an asynchronous context and outlines a distributed diffusion-based and a semi-static load balancing algorithm. Lastly, in Section \ref{sec:experiments}, we present our model validation and experimental results, demonstrating the viability of these load balancing approaches.
