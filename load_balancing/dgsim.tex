\section{The DGSim Simulator}
\label{sec:dgsim}
Our DGSim simulator utilizes discrete event simulation to model the
execution of the parallel application.
It is expected to only run {\em skeletonized} applications, where heavy
computation and large data allocations are omitted for efficiency.
Every thread in the simulation schedules events into a global priority queue
keyed on {\em virtual time}, so that events are processed in the
correct simulation order.
Threads ``burn'' virtual time to simulate the execution of heavy computational
tasks, and message arrival times are delayed to capture communication costs.
This method permits efficient large-scale
simulation while retaining the salient computation and communication
characteristics of the program execution.
With DGSim, we are able to rapidly evaluate hundreds of simulation trials with different
parameterizations in 4,000 core-hours compared to over 21,000,000 core-hours
had we run DGSWEM itself, roughly translating to a 5,200x speed-up.

% (i.e. "burn" 10,000 cycles of time here, or send a 64KB message there).

DGSim is designed from the ground up to model a very aggressive
asynchronous execution framework, which is comprised of three layers.
%The lowest layer embodies a traditional distributed 
%parallel machine consisting of multi-threaded processes communicating 
%via messaging. Above that sits the active global address space (AGAS) 
%for managing locality. Finally, the layer to which the application 
%binds itself is that of a dataflow tasking model.
The lowest layer of our framework, the parallel 
machine, mirrors the communicating multi-threaded processes prevalent 
in current extreme scale computing.
%Our targeted configuration is the 
%fat-process model, where there are relatively few processes per node,
%but each contains threads dedicated to the physical cores.
All inter-processes communication is expressed through one-sided,
point-to-point active messages, each containing a bound function.
% (code address + all needed arguments).
%All processes are assumed to be running the same executable 
%binary so that any function available to the sender will be known by 
%the recipient. Upon receipt of a message, the destination process 
%simply executes the given function against the arguments. Sending 
%messages and polling for incoming messages are the only provided 
%communication primitives. The semantics of sending a message are that 
%the message will eventually be executed by the recipient so long as 
%they periodically poll for incoming messages, but with no ordering 
%guarantees. 
%As a result, the only way for a sender to be notified of a message's 
%delivery is for the recipient to explicitly send a message back. 
%This 
%highlights the very asynchronous nature of active messages: they 
Active messages impose no synchronization, and thus encourage a very
asynchronous methodology.
%The threads within a process are intended to execute in the coherent 
%shared memory programming model common today, but to avoid having to 
%track individual loads and stores in our simulator, we've chosen to 
%restrict and abstract how threads may signal each other. 
Each process contains a {\em master} thread responsible for executing
active messages, and all other threads belong to a pool of {\em worker} threads.
%There is a single-producer 
%multi-consumer active message queue with which the master pushes 
%messages for an arbitrary worker to execute, and likewise, a 
%multi-producer single-consumer queue to which workers push messages 
%that the master executes.
Messages from other processes are only 
executed by the master thread, but any thread may send messages to 
other processes.
%which it can expect to have injected into the network 
%concurrently with respect to other sends leaving that process.
%Our threading model's restrictive nature is severe in that there are 
%innumerably many applications that do not manage their threads this 
%way. But we feel that this design point is valid in the context of 
%asynchronous execution. The message queues connecting threads are 
%non-synchronizing and hence do not block producers which allows for 
%pipelining of work between master and workers. Also, 
The dedication of the master thread to handling
incoming messages and managing worker threads may seem wasteful,
but this fits asynchronous 
execution well as it ensures there is a core that will remain 
attentive to servicing incoming messages.
Production asynchronous runtimes such as Charm++
usually dedicate at least one CPU core for communication and scheduling.
%We believe 
%this is a crucial factor for distributed dynamic scheduling algorithms 
%where processes are continuously probing the status of their peers.
%Notable production asynchronous runtimes such HPX and Charm++ also usually
%provision at least one CPU core to the task of handling communication
%and/or scheduling for this reason, thus validating the merit of our design.

The second layer of the framework's stack is our interpretation of 
Active Global Address Space (AGAS) \cite{hpx2}. This is a software layer
that allows the user to place parts of their application state in a 
named-object store without having to explicitly manage where
objects reside. Users simply {\tt visit} objects by sending an active 
message to a name (as opposed to a process id) and the function will be 
executed on the process currently holding the object.
To migrate an object, there is a {\tt relocate} function that takes a 
name and a target process id and will cause the named object to 
migrate from wherever it currently resides to the target.
Relocates and visits can be issued concurrently from any 
number of processes without any synchronization whatsoever.
%AGAS guarantees that 
%visits will track down even the most frantically relocating objects. 
%The AGAS implementation is pure in the sense that it receives no 
%special attention from the simulator. It is built solely on active 
%messages as a distributed hash table spanning the processes to track 
%and cache object locality. We note that at this point, our design comes
%quite close 
This design is similar to the chare/actor model of Charm++.
They too express parallel computation as asynchronous method invocations
between relocatable objects.

The top layer is a distributed tasking layer, which allows
the application to describe named units of non-blocking work that are executed 
on worker threads.
%The application may 
%attach priority values to tasks to influence the order satisfied tasks 
%are executed.
%They are non-blocking, meaning that once begun, 
%a task can not wait for any event dependent on the execution of another 
%task.
Tasks mainly communicate via satisfaction of another task's dependencies.
%, but they may not 
%poll for incoming messages (as they are forbidden by virtue of 
%executing on a worker thread).
The task graph has no explicit hierarchy 
(no parent or sub-tasks), and all dependencies are managed through task 
names. When a task has computed data required by other tasks, that task 
sends a satisfaction containing the data to its successors.
%A satisfaction consists of a 
%task name, a credit count, and an active message. When the task runtime 
%sees a task name for the first time, a new task is instantiated. Upon 
%instantiation, a task must know, as a function of its name, its initial 
%credit count. The receipt of a satisfaction causes the active message 
%payload to be executed upon the task's state (to deliver required 
%data), and the task's credit count to be decremented by the 
%satisfaction's credit value. If a task's credits reach zero, the task 
%is considered satisfied and will be pushed to a worker thread for 
%execution after which the task will be destroyed. 
%Thanks to good engineering of our layers, 
All task satisfactions are sent through AGAS {\tt visit}s, allowing 
task migrations to be issued independently of the task graph scheduler.
%This makes the exploration of schedulers tractable 
%as locality decisions and task management are decoupled units of 
%software, and the 
This separation allows the scheduling logic to reside entirely in the
application code, giving it easy access to pertinent metadata, and
enables task migration to occur concurrently with task execution.
Thus we have no need to explicitly enter or exit a load balancing phase.
%There is no need to enter or exit a load balancing phase -- task
%migrations will occur completely behind the scenes.
