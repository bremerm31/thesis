\section{Conclusion}

This work presented an adaptive local timestepping algorithm for conservation laws. Loop invariants were used to derive a proof of formal correctness, ensuring that the algorithm is total variation stable even when considering dynamic changes in local wave speeds. Furthermore, the algorithm was parallelized using a speculative parallel discrete event simulator. Results indicate that the parallelization recovers a majority of the expected speed-up, when accounting for the cost of the CFL condition.

As part of our future work, we will examine higher order timestepping strategies. The forward Euler step forms the basis of higher order strong stability preserving methods. Future work will include the development of higher order adaptive timestepping methods in the spirit of~\cite{Constantinescu2007}. Furthermore, we aim to apply this method to physically relevant coastal modeling problems to better quantify benefit of adaptive local timestepping to communities of interest.

At a more general level this work makes two important contributions to the development of asynchronous algorithms. Firstly, the nonlinearity of this problem makes knowing the task graph a priori impossible. While there has been an emergence of task-based runtimes in recent years to address architectural heterogeneity, this work calls into question the efficacy of a task-dependency graph as the sole means of representing scientific computing work flows within these modern runtime systems. Close interdisciplinary work is required with runtime system engineers and applied mathematicians to ensure that avenues for efficient implementations of adaptive algorithms remain accessible.

Secondly, the application of loop invariant analysis in order to assess the correctness of application significantly simplifies the development. Formal correctness techniques are expected to play an increasingly important role in the era of extreme heterogeneity as exhaustively replicating the states of a system in a highly concurrent setting will become increasingly difficult~\cite{ExtremeHeterogeneity2018}. The results here showcase how to use the invariant analysis proposed in~\cite{Bientinesi2011}. Due to the formal correctness proof, we ran into significantly fewer bugs during application development. Nevertheless, we found the derivation of the proof to be tedious. Automating this proof technique would allow us to better manage complexity and support the derivation of increasingly complicated algorithms.

