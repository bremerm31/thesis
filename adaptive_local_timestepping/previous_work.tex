\section{Previous Work}
\label{sec:prev}

Relaxing the CFL condition for conservation laws by allowing different regions of the mesh to advance with different timesteps has been the subject of numerous studies. Osher et al. presented a first-order total variation diminishing timestepping scheme in~\cite{Osher1983}. Dawson and Kirby developed a second order method, which reduces to first order at the interface between elements stepping with different timesteps~\cite{Dawson2000}. Kirby generalized this approach to high resolution methods in~\cite{Kirby2003}. Constantinescu developed a second order total variation stable method for conservation laws~\cite{Constantinescu2007}, leveraging the P-series analysis of Hairer~\cite{Hairer1981}. Third order methods were presented in the work of Schlegel et al~\cite{Schlegel2009}. These methods can conserve mass, however are unable to demonstrate total variation stability.

Other methods have relied on high-order interpolation strategies to approximate coupled terms with larger timesteps~\cite{Krivodonova2010, Gupta2016, Hoang2019}. Another set of local timestepping algorithms for conservation laws rely on single step methods with high-order space-time representations~\cite{Loercher2007,Dumbser2007,Taube2009}. These methods are conservative and high-order, however fail to recover the total variation stability results of the partitioned Runge-Kutta approaches.

The main thrust of this chapter is a reformulation of the algorithm of Osher et al~\cite{Osher1983} to improve parallel performance. Previous work parallelizing local timestepping approaches has had mixed results. In cases, where the CFL number is fixed, e.g. for linear conservation laws with static meshes, task-based programming models have been extremely effective, with local timestepping ADER-DG methods scaling up to full system runs~\cite{Breuer2016,Uphoff2017}. 

While task-based approaches have proven successful for linear conservation laws, their implementations fundamentally rely on timestepping groups being statically determined, and thus are unsuitable for timestepping adaptivity.
Ignoring dynamic changes in timestepping groups results in CFL violations and associated instabilities, e.g.~\cite{Trahan2012,Gnedin2018}. 
Other parallelization attempts for conservation laws have been made by recursively updating finer and finer timestepping groups. These methods would allow for adaptive timestepping by synchronizing after each fine timestep to allow for cells to change their timestepping groups. However, load balancing these methods requires balancing the work in each timestepping group across all ranks~\cite{Rietmann2015,Seny2014}. The scalability of these approaches is limited by the amount of work in the timestepping group with the smallest timestep.
Other parallelization approaches have treated the adaptive local timestepping problem as a discrete event simulation~\cite{Karimabadi2006, Omelchenko2006, Unfer2007, Omelchenko2012, Omelchenko2014,Stone2017,Shao2019}. While these approaches achieve tremendous speed-ups, these methods typically provide only heuristics and examples as proofs of robustness. To the authors' knowledge, no method has presented a provably total-variation stable adaptive timestepping scheme.