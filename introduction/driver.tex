\chapter{Introduction}% 1 page overview

% why hurricane simulations are important
%Hurricane events are incredibly deadly and costly natural disasters.
Since 1980, seven out of the ten most costly US climate disasters were hurricanes, with Hurricane Katrina being the most expensive\cite{NCEI2018}. Hurricanes Harvey, Maria, and Irma, which occurred in 2017, are expected to be among the five most costly US natural disasters.
The utilization of computational models can provide local officials with high fidelity tools to assist in evacuation efforts and mitigate loss of life and property.
%why we need HPC
Due to the highly nonlinear nature of hurricane dynamics and stringent time constraints, high performance computing (HPC) plays a cornerstone role in providing accurate predictions of flooding.
Because of the importance of fast, efficient models, there is a significant interest in improving the speed and quality of these computational tools.

%how computing is changing
Even as the speed of supercomputers is drastically increasing, the end of Moore's law and the introduction of many-core architectures represent a tectonic shift in the HPC community.
In particular, the degree of hardware parallelism is increasing at an exponential rate and the cost of data movement and synchronization is increasing faster than the cost of computation, and hardware is becoming increasingly irregular due to the use of accelerators and susceptibility to failure~\cite{Kogge2013}.
In order to achieve good resource utilization on these machines, task-based programming and execution models are being developed to express increased software parallelism, introduce more flexible load balancing capabilities, and hide the cost of communication through task over-decomposition. 
Examples of major task-based programming and execution models include Charm++, HPX~\cite{hpx2}, Legion~\cite{legion}, OCR~\cite{ocr}, PaRSEC~\cite{parsec}, StarPU~\cite{starpu}, etc.
There are also domain-specific task-based programming systems, such as the Uintah AMR Framework~\cite{uintah}, and task-based portability layers such as DARMA~\cite{darma}.
% Additionally, task-based execution may help address issues such as hardware fault tolerance.
These programming models decouple the specification of the algorithm from the task scheduling mechanism, which determines where and when each task may execute and orchestrates the movement of required data between the tasks.
Furthermore, lightweight, one-sided messaging protocols that support active messages have the potential to reduce the overheads associated with inter-process communication and synchronization, which will become even more important as parallelism increases.
%However, transitioning scientific applications from their current synchronous implementations to asynchronous, task-based programming models requires a significant software engineering effort.
%Additionally, load balancing the execution of the resulting task graph remains a challenge due to the underlying computational hardness of optimal task scheduling.
% Often, load balancers must utilize application-specific information to obtain a reasonable schedule.

The aim of this thesis is to explore the co-design of algorithms for hurricane storm surge with task-based runtime systems. While simply porting algorithms will certainly allow us to mitigate the impact of hardware irregularity, we hafve found that in practice this gives at best a marginal improvement over existing optimized MPI implementations. 

In~\cite{Bremer2019}, we found that HPX---one such runtime---gives a speed-up of 1.2 over an MPI implementation on 128 Knights Landing nodes. Results suggest that the speed-up was mainly attributed to page faults for the MPI implementation, rather than avoiding message latencies or MPI overhead. In fact, we suspect that this gap could largely be closed through use of a better memory allocator for the MPI implementation. 

%Furthermore, the HPX version is able to strong scale substantially less well than MPI version. Extrapolating results from \cite{Bremer2019}, a 4 day forecast using a DG method would require 17.5 hours at the strong scaling limit.  The required minimum task sizes for good runtime efficiency (i.e. fraction of time spent in the application versus the runtime) puts the HPX implementation well outside of the scalability regime needed for real-time forecasting of storm surge.

The test case used in~\cite{Bremer2019} ultimately lacks irregularity. The communication profile of the code reduces to a stencil code on an unstructured mesh, which can be implemented efficiently using non-blocking point to point MPI messages. What is needed to generate a value proposal for using task-based runtime systems are sources of irregularity. This thesis is split into two main chapters, identifying and assessing the impact of two such ideas:
\begin{enumerate}
\item {\em Chapter \ref{ch:lb}}: An essential part to the simulation of hurricane storm surge is the simulation of coastal flooding. One existing source of irregularity is the classification of cells as either wet or dry. Wet cells require the full set of physics to be solved, while dry cells are able to trivially update. In this chapter, we explore the impact of dynamic load balancing strategies to address the load imbalance generated as a hurricane makes land fall.
\item {\em Chapter \ref{ch:lts}}: The timestep taken by these simulations is restricted by the Courant-Friedrichs-Lewy (CFL) condition. The CFL condition forms the basis of the stability results which have led to the popularity of finite volume and DG methods. However, this condition is ultimately a condition that can be enforced locally, and in light of significant variations in mesh size and advection speeds, local CFL enforcement can lead to dramatic reduction in work. This chapter develops a timestepping method that locally enforces the CFL and recovers the same stability results for the local timestepping case.
\end{enumerate}


\section{Storm Surge Modeling}
The motivating application for this work is the numerical simulation of large-scale coastal ocean physics, in particular, the modeling of hurricane storm surges.  One of the leading simulation codes in this area is the Advanced Circulation (ADCIRC) model, developed by a large collaborative team including some of the co-authors \cite{ADCIRC2017,Westerink2008,Bunya2010,Dietrich2010,Dietrich2011a,Dietrich2011b,Hope2016,Hope2013}.  ADCIRC is a Galerkin finite element based model that uses continuous, piecewise linear basis functions defined on unstructured triangular meshes.  The model has been parallelized using MPI and has been shown to scale well to a few thousand processors for large-scale problems \cite{Tanaka2011}.   While ADCIRC is now an operational model within the National Oceanographic and Atmospheric Administration's Hurricane Surge On-Demand Forecast System (HSOFS),  its performance on future computational architectures is dependent on potentially restructuring the algorithms and software used within the model.  Furthermore, ADCIRC provides a low-order approximation and does not have special stabilization for advection-dominated flows, thus requiring a substantial amount of mesh resolution.  Extending it to higher-order or substantially modifying the algorithms within the current structure of the code is a challenging task.

With this in mind, our group has also been investigating the use of discontinuous Galerkin (DG) methods for the shallow water equations \cite{Kubatko2006,Kubatko2009a,Kubatko2007,Bunya2009,Wirasaet2014,Wirasaet2015,Michoski2017,Michoski2015,Michoski2013,Michoski2011}, focusing on the Runge-Kutta DG method as described in \cite{Cockburn1989a}.  We have shown that this model can also be applied to  hurricane storm surge \cite{Dawson2011}.  DG methods have potential advantages over the standard continuous Galerkin methods used in ADCIRC, including local mass conservation, ability to dynamically adapt the solution in both spatial resolution and polynomial order ($hp$-adaptivity), and potential for more efficient distributed memory parallelism \cite{Kubatko2009b}.  While DG methods for the shallow water equations have not yet achieved widespread operational use, recent results have shown that for solving a physically realistic coastal application at comparable accuracies, the DG model outperformed ADCIRC in terms of efficiency by a speed-up of 2.3 and when omitting eddy viscosity, a speed-up of 3.9~\cite{Brus2017}.

The prediction of hurricane storm surge involves solving physics-based models that determine the effect of wind stresses pushing water onto land and the restorative effects of gravity and bottom friction. These flows typically occur in regimes where the shallow water approximation is valid \cite{Dutykh2016,Michoski2013}   . Taking the hydrostatic and Boussinesq approximations, the governing equations can be written as
%\Kazbek{colon after "as"}
\begin{align*}
\partial_t \zeta + \nabla \cdot \mathbf{q} &= 0,\\
\partial_t q_x + \nabla \cdot ( \mathbf{u}q_x) + \partial_x g(\zeta^2/2 + \zeta b) &= g \zeta \partial_x b + S_1,\\
\partial_t q_y + \nabla \cdot ( \mathbf{u}q_y) + \partial_y g( \zeta^2/2 + \zeta b) &= g \zeta \partial_y b + S_2,
\end{align*}
where:
\begin{itemize}
\item $\zeta$ is the water surface height above the mean geoid,
\item $b$ is the bathymetry of the sea floor with the convention that downwards from the mean geoid is positive,
\item $H = \zeta + b$ is the water column height,
\item $\vec{u} =\left[ u \, , \, v\right]^T$ is the depth-averaged velocity of the water,
\item $\vec{q} = H \vec{u} = \left[ q_x \, , \, q_y \right]^T$ is the velocity integrated over the water column height.
\end{itemize}
Additionally, $g$ is the acceleration due to gravity, and $S_1$ and $S_2$ are source terms that introduce additional forcing associated with relevant physical phenomena, e.g. bottom friction, Coriolis forces, wind stresses, etc. %\Craig{Still think not posting units is a mistake)}

\section{The Discontinuous Galerkin Finite Element Method}

The discontinuous Galerkin (DG) kernel originally proposed by Reed and Hill~\cite{Reed73} has achieved widespread popularity due to its stability and high-order convergence properties. For an overview on the method, we refer the reader to~\cite{Cockburn01,Hesthaven08} and references therein. For brevity, we forgo  rigorous derivation of the algorithm, but rather aim to provide the salient features of the algorithm to facilitate discussion of the parallelization strategies.

We can rewrite the shallow water equations in conservation form
\begin{equation}
\partial_t \mathfrak{U} + \nabla \cdot\vec{F}(t,\vec{x}, \mathfrak{U}) = S(t,\vec{x},\mathfrak{U}),
\label{eq:conslaw}
\end{equation}
where
\begin{equation*}
\mathfrak{U} = \mat \zeta \\ q_x \\ q_y \rix, \quad \vec{F} = \mat q_x && q_y \\ u^2H + g(\zeta^2/2 + \zeta b) && uv H \\
uv H && v^2 H + g ( \zeta^2 /2 + \zeta b) \rix.
\end{equation*}

Let $\Omega$ be the domain over which we would like to solve Equation \eqref{eq:conslaw}, and consider a mesh discretization $\Omega^h = \cup_e^{n_{el}} \Omega_e^h$ of the domain $\Omega$, where $n_{el}$ denotes the number of elements in the mesh.

We define the discretized solution space, $\Wh$ as the set of functions such that for each state variable the restriction to any element $\Omega_e^h$ is a polynomial of degree $p$.  Note that we enforce no continuity between element boundaries, which is the key feature enabling high-order convergence in discontinuous Galerkin methods.  Let $\ip{f}{g}_{\Gamma} = \int_{\Gamma} fg \diff x$ denote the $L^2$ inner product over a set $\Gamma$. The discontinuous Galerkin formulation then approximates the solution by projecting $\mathfrak{U}$ onto $\Wh$ and enforcing Equation \eqref{eq:conslaw} in the weak sense over $\Wh$, i.e.
\begin{equation*}
\ip{ \partial_t \vec{U} + \nabla \cdot F(t,\vec{x},\vec{U}) - S(t,\vec{x}, \vec{U})}{\vec{w}}_{\Omega^h} = 0
\end{equation*}
for all $\vec{w} \in \Wh$, where $\vec{U} \in \Wh$ denotes the projected solution.  %\Craig{I find this description a bit needlessly abstruse. Can you make it a bit more accessible, particularly to people who aren't FEM specialists?}
%\Max{Define $\vec{U}$ as the numerical solution.}
Since the indicator functions over each element are members of $\Wh$, it suffices to satisfy the weak formulation element-wise.
The discontinuous Galerkin method can be alternatively formulated as
\begin{equation}
\partial_t \ipelt{\vec{U}}{\vec{w}} = \ipelt{\vec{F}}{\nabla\vec{w}} - \ipedg{\widehat{\vec{F}\cdot \vec{n}}}{\vec{w}} + \ipelt{\vec{S}}{\vec{w}}
\label{eq:dg}
\end{equation}
for all $\vec{w} \in \bigoplus_{d=1}^3 \mathcal{P}^p(\Omega_e^h)$ and for all $e = 1, \ldots, n_{el}$, where $\mathcal{P}^p(\Omega_e^h)$ is the space of polynomials of degree $p$ over $\Omega_e^h$. Due to the discontinuities between elements in both trial and test spaces, particular attention must be given to the boundary integral term, which is not well-defined even in a distributional sense. For evaluation, the boundary integral's integrand is replaced with a numerical flux $\widehat{\vec{F} \cdot \vec{n}}(\vec{U}^{int}, \vec{U}^{ext}) \vec{w}^{int}$. To parse this term, let $\vec{U}^{int}$ and $\vec{w}^{int}$ denote the value of $\vec{U}$ and $\vec{w}$ at the boundary taking the limit from the interior of $\Omega^h_e$, and let $\vec{U}^{ext}$ denote the value of $\vec{U}$ at the boundary by taking the limit from the interior of the neighboring element. For elements along the boundary of the mesh, the boundary conditions are enforced by setting $\vec{U}^{ext}$ to the prescribed values. For the numerical flux, $\widehat{\vec{F}\cdot\vec{n}}$, we use the local Lax-Friedrichs flux, 
\begin{equation*}
\widehat{\vec{F} \cdot \vec{n}}(\vec{U}^{int}, \vec{U}^{ext}) = \frac{1}{2} \left( \mathbf{F}(\vec{U}^{int}) + \mathbf{F}(\vec{U}^{ext}) + |\Lambda| (\vec{U}^{ext} - \vec{U}^{int}) \right) \cdot \mathbf{n},
\end{equation*}
where $\mathbf{n}$ is the unit normal pointing from $\Omega^h_e$ outward, and $|\Lambda|$ denotes the magnitude of the largest eigenvalue of $\nabla_u \mathbf{F} \cdot \vec{n}$ at $\vec{U}^{int}$ or $\vec{U}^{ext}$.

In order to convey more clearly the implementation of such a kernel in practice, consider the element $\Omega_e^h$. For simplicity of notation for the remainder of the subsection we drop all element-related subscripts, $e$. Over this element, we can represent our solution using a basis, $\{ \varphi_i\}_{i=1}^{n_{dof}}$. Then we can let our solution be represented as
\begin{equation*}
\vec{U}(t,x) = \sum_{i=1}^{n_{dof}} \coeffU_i(t) \varphi_i(x),
\end{equation*}
where $\coeffU$ are the basis-dependent coefficients describing $\vec{U}$.
Following the notation of Warbuton~\cite{Gandham2015}, it is possible to break down Equation \eqref{eq:dg} into a set of kernels as 
\begin{equation}
\partial_t \coeffU_i = \sum_{j=1}^{n_{dof}} \mathcal{M}_{ij}^{-1}\left( \underbrace{\ipelt{\vec{F}}{\nabla\varphi_j}}_{\mathcal{V}_{j}}  + \underbrace{\ipelt{\vec{S}}{\varphi_j}}_{\mathcal{S}_{j}} - \underbrace{\ipedg{\widehat{\vec{F}\cdot \vec{n}}}{\varphi_j}}_{\mathcal{I}_{j}} \right),
\end{equation}
where $\mathcal{M}_{ij} = \ipelt{\varphi_i}{\varphi_j}$ denotes the local mass matrix.
Here we define the following kernels:
\begin{itemize}
\item $\mathcal{V}$: The volume kernel,
\item $\mathcal{S}$: The source kernel,
\item $\mathcal{I}$: The interface kernel.
\end{itemize}
To discretize in time, we use the strong stability preserving Runge-Kutta methods~\cite{Gottlieb2001}. Letting 
\begin{equation*}
\mathcal{L}^h \left(\coeffU \right) = \mathcal{M}^{-1}\left( \mathcal{V}\left(\coeffU\right) + \mathcal{S}\left(\coeffU\right) - \mathcal{I}\left(\coeffU\right) \right),
\end{equation*}
we can define the timestepping method, for computing the $i$-th stage as
\begin{equation*}
\coeffU^{(i)} = \sum_{k=0}^{i-1} \alpha_{ik} \coeffU^{(k)} + \beta_{ik} \Delta t \mathcal{L}^h\left(\coeffU^{(k)}\right),
\end{equation*}
where $\coeffU^{(k)}$ denotes the basis coefficients at the $k$-th Runge-Kutta stage.
We denote the operator, which maps $\left\{\coeffU^{(k)},\,\mathcal{V}\left(\coeffU^{(k)}\right),\,\mathcal{S}\left(\coeffU^{(k)}\right),\,\mathcal{I}\left(\coeffU^{(k)}\right)\right\}_{k=0}^{i-1}$ to $\coeffU^{(i)}$ as the update kernel, $\mathcal{U}$.

Solutions to hyperbolic conservation laws may give rise to discontinuous solutions. In these regions, the underlying approximation theory breaks down and the DG algorithm gives rise to spurious oscillations, which if left untreated give rise to spurious oscillations. These oscillations are treated in a post processing phase known as {\em slope-limiting}. Techniques roughly are categorized as limiting based on neighboring information \cite{Cockburn1989b,Kuzmin2010,Bell1988}, reducing high frequency modes via filtering \cite{Hesthaven08,Maday1993,Meister2011}, adding viscosity to smear out sharp gradients \cite{Guermond2011,Persson2006}, or projection to lower orders \cite{Dumbser2016}. For an in-depth comparison of various slope limiting approaches, we refer the reader to~\cite{Michoski2015}. For the shallow water equations, an additional instability occurs for small water column heights. The numerics may give rise to regions of negative water column height causing the shallow water equations to become meaningless both physically and mathematically. To address this, numerous approaches have been proposed~\cite{Bunya2009,Gandham2015,Casulli2009,Vater2015,Xing2013,Rannabauer2018}.

\section{Exascale Computing--Novel Programming Models} % 3 pages HPC/DLB
Even as the speed of supercomputers is drastically increasing, the end of Moore's law and the introduction of many-core architectures represent a tectonic shift in the HPC community.
In particular, the degree of hardware parallelism is increasing at an exponential rate and the cost of data movement and synchronization is increasing faster than the cost of computation~\cite{Kogge2013}.
% , and hardware is becoming increasingly irregular due to the use of accelerators and susceptibility to failure.
% In order to achieve good resource utilization on these machines, task-based programming and execution models are being developed to
As a result, asynchronous task-based programming has been of great interest due to its potential to express increased software parallelism, introduce more flexible load balancing capabilities, and hide the cost of communication through task over-decomposition.
Examples of major task-based programming and execution models include Charm++~\cite{charm++}, HPX~\cite{hpx2}, Legion~\cite{legion}, OCR~\cite{ocr}, PaRSEC~\cite{parsec}, StarPU~\cite{starpu}, etc.
There are also domain-specific task-based programming systems, such as the Uintah AMR Framework~\cite{uintah}, and task-based portability layers such as DARMA~\cite{darma}.
% Additionally, task-based execution may help address issues such as hardware fault tolerance.
These programming models decouple the specification of the algorithm from the task scheduling mechanism, which determines where and when each task may execute and orchestrates the movement of required data between the tasks.
Furthermore, lightweight, one-sided messaging protocols that support active messages have the potential to reduce the overheads associated with inter-process communication and synchronization, which will become even more important as parallelism increases.

\subsubsection{Active Global Address Space}

The most relvant abstraction with respects to load balancing is the active global address space (AGAS). As we intend to implement dynamic load balancing using an HPX framework, we describe the AGAS here.
While the threading subsystem operates within a single private address space, HPX extends its programming model to distributed runs via an \emph{active global address space}. Global address spaces attempt to emulate the ease of programming on a single node, while still maintaining tight control over data locality necessary for writing a performant distributed code. Two well-known global address space models are UPC~\cite{upc} and Co-Array Fortran~\cite{coarray}. While global address space models like UPC's partitioned global address space (PGAS) are more data-centric, e.g. by exposing pointers to memory addresses on different nodes, HPX approaches global address spaces in a more object-oriented manner.

AGAS consists of a collection of private address spaces, called \emph{localities}. Each locality will run its own instance of the threading subsystem, scheduling threads with locally available resources. In practice, localities are typically chosen to be  nodes or NUMA domains. The basic addressable unit in AGAS is the \emph{component}. Components encapsulate the objects the user would like to remotely access. To interact with a component, the user must go through a smart-pointer-like wrapper class called a \emph{client}. The client can not only manage the component's lifetime via the RAII idiom, but also exposes remotely invokable member functions. Clients can either reside on the same or different locality as their associated component. When a remote locality executes a client member function, HPX will send an active message to the locality where the component is located, execute the function there, and return the result to the client. By interfacing with components through clients, AGAS provides equivalent local and distributed semantics, simplifying the programming of distributed applications.

The key difference between AGAS and other global address space models is its native support for component \emph{migration}. HPX's AGAS layer allows the developer to relocate components to different localities during runtime. With all component functions being invoked through the client interface, HPX is able to ensure that the component functions invoked through the client will be executed on the correct localities, guaranteeing the application's correctness. This functionality can be used to accelerate applications via dynamic load balancing. However, since component migration potentially requires sending large quantities of data through the interconnect, and the load profile is application dependent, HPX requires the application to manage component relocation.

AGAS decouples the correctness of the application from the data distribution across nodes.  This abstraction provides a means of ensuring load balance across nodes without imposing unreasonable burdens on the application to guarantee correctness.

\subsubsection{Dynamic Load Balancing}
One of the key aspects of DGSWEM is its ability to simulate coastal inundation during hurricane landfall.
The DG kernel can be implemented in a manner such that dry regions of the simulation require no computational work; however, as the hurricane inundates the coast, this optimization introduces significant dynamic load imbalance.
In order to address the load imbalance while still fitting the problem in machine memory, two constraints (load and memory) must be accounted for simultaneously.

While multi-constraint graph partitioning tools have been used to obtain good static partitions for these scenarios, they perform sub-optimally for irregular applications such as ours.

In adaptive mesh refinement (AMR) codes and $hp$-adaptive finite elements, block-structured grids are commonly used, which allow for efficient representations of data locality using space filling curves. These space filling curves allow for dynamic load balancing techniques, known as geometric partitioning and outlined in \cite{Devine2005,Burstedde2011,Blaise2012,Ferreira2017}. However, for our purposes block-structured grids have difficulties modeling complex coastal geometries. As such, DGSWEM utilizes unstructured grids. Dynamic load balancing for these grids typically relies on two approaches: (1) graph partitioning algorithms, such as those provided in the METIS and SCOTCH libraries \cite{scotch,Bhatele2012,Karypis1998,Devine2005,Aykanat2007}, and (2) diffusion or refinement based approaches \cite{Schloegel1997}.
The simulation of coastal inundation introduces irregularity, which decouples the memory and load balancing constraints.
%, i.e. balancing elements across processors is not sufficient
To the authors' knowledge, the only paper that uses dynamic load balancing to address this issue is \cite{Asuncion2016}. However, their approach only balances load on structured grids, which may result in memory overflow.
Local timestepping methods introduce similar irregularity.
Seny et al. have proposed a static load balancing scheme using multi-constraint partitioning in \cite{Seny2014}.
However, they note the dynamic load balancing problem as an open one.
Some examples of load balancing algorithm evaluations in the context of task-based execution models include the use of cellular automata \cite{Hosoori2011}, hierarchical partitioning \cite{Zheng2010}, and gossip protocols \cite{Menon2013}.

%\subsection{Local Timestepping} % 5 pages Local Timestepping
%\input{introduction/local_timestepping}